\documentclass[aip,jcp,reprint,noshowkeys,superscriptaddress]{revtex4-1}
\usepackage{graphicx,dcolumn,bm,xcolor,microtype,multirow,amsmath,amssymb,amsfonts,physics,mhchem,xspace,subfigure}

\usepackage[utf8]{inputenc}
\usepackage[T1]{fontenc}
\usepackage{txfonts}

\usepackage[
	colorlinks=true,
    citecolor=blue,
    breaklinks=true
	]{hyperref}
\urlstyle{same}

\definecolor{darkgreen}{HTML}{009900}
\usepackage[normalem]{ulem}
\newcommand{\sphi}[1]{\hat{{\bf S}}_{#1}}
\newcommand{\overlap}[2]{\langle #1 | #2 \rangle}
\newcommand{\matelem}[3]{\langle #1 | #2 | #3 \rangle}
\newcommand{\deriv}[3]{\frac{\partial^{#3} #1}{\partial {#2}^{#3}}}
\newcommand{\bd}[1]{{\bf {#1}}}
\newcommand{\br}[0]{{\bf {r}}}
\newcommand{\bs}[0]{{\bf {s}}}
\newcommand{\dr}[1]{\text{d}{\bf {#1}}}
\newcommand{\psiex}[0]{\Psi^{\text{ex}}}
\newcommand{\energyex}[0]{E^{\text{ex}}}
\newcommand{\opr}[1]{\,\hat{#1}}
\newcommand{\oprd}[1]{\,\hat{#1}^\dagger}

\begin{document}	

\title{Computing matrix element of the three-body operator for parallel spins}

\author{Emmanuel Giner}
\email{emmanuel.giner@lct.jussieu.fr}

\maketitle
\section{What we need to compute}
\subsection{General matrix element}
We need to compute the matrix elements of this nasty operator
\begin{equation}
 O^3 = \frac{1}{6} \sum_{ijm} \sum_{kln} \overlap{ijm}{kln} \oprd{k} \oprd{l} \oprd{n}  \opr{m} \opr{j} \opr{i}.
\end{equation}
Therefore, it is the product of $\oprd{k} \oprd{l} \oprd{n}$ by $\opr{m} \opr{j} \opr{i}$, and each term posses a 6-fold permutation symmetry with $\pm 1$ signature. 
Starting from $\oprd{k} \oprd{l} \oprd{n}$ there are two symmetric pertumations 
\begin{equation}
 \begin{aligned}
   & \oprd{k} \oprd{l} \oprd{n} = \oprd{l} \oprd{n} \oprd{k} = \oprd{n} \oprd{k} \oprd{l}, \\
 \end{aligned}
\end{equation}
and three anti-symmetric permutations 
\begin{equation}
 \begin{aligned}
  \oprd{l} \oprd{k} \oprd{n}  =   \oprd{n} \oprd{l} \oprd{k} =  \oprd{k} \oprd{n} \oprd{l},
 \end{aligned}
\end{equation}
which fulfill 
\begin{equation}
 \begin{aligned}
 \oprd{k} \oprd{l} \oprd{n} = - \oprd{l} \oprd{k} \oprd{n}. 
 \end{aligned}
\end{equation}
Therefore, one can express a $6\times 6$ matrix that is the product all possible permutations for $(k,l,n)$ and $(i,j,m)$, rearrange everything in terms of the original $\oprd{k} \oprd{l} \oprd{n}  \opr{m} \opr{j} \opr{i}$ operator and obtain the following terms 
\begin{equation}
 \begin{aligned}
 & \frac{1}{6} \mathcal{P}_{ijm}^{kln}\overlap{ijm}{kln} \oprd{k} \oprd{l} \oprd{n}  \opr{m} \opr{j} \opr{i}= &\oprd{k} \oprd{l} \oprd{n}  \opr{m} \opr{j} \opr{i} \bigg( \\
 &+ \big( \overlap{ijm}{kln} + \overlap{ijm}{lnk}  + \overlap{ijm}{nkl} \big) \\
 &- \big( \overlap{ijm}{lkn} + \overlap{ijm}{nlk}  + \overlap{ijm}{knl} \big) \bigg),
 \end{aligned}
\end{equation}
where $\mathcal{P}_{ijm}^{kln}$ contains the sum over the 36 permutations of the 6 indices with the corresponding signs. 

\subsection{Diagonal term}
For a diagonal term, one has that 
\begin{equation}
 \begin{aligned}
  i = k \\
  j = l \\
  m = n \\
 \end{aligned}
\end{equation}
and therefore one obtains 
\begin{equation}
 \begin{aligned}
 \bra{I}O^3\ket{I} = \sum_{i>j>m\in \ket{I}}&+ \big(  \overlap{ijm}{ijm} + \overlap{ijm}{jmi}  + \overlap{ijm}{mij} \big) \\
                                 &- \big(  \overlap{ijm}{jim} + \overlap{ijm}{mji}  + \overlap{ijm}{imj} \big)
 \end{aligned}
\end{equation}
where the summation $\sum_{i>j>m\in \ket{I}}$ means running over all orbitals occupied in $\ket{I}$ in a unique way. 
\subsection{Single excitation term}
For a single excitation term, one has that 
\begin{equation}
 \begin{aligned}
  i \ne k \\
  j = l \\
  m = n \\
 \end{aligned}
\end{equation}
and then 
\begin{equation}
 \begin{aligned}
 \bra{I}O^3 \oprd{k} \opr{i} \ket{I} = (-1)^{\mathcal{P}_{IJ}}
 \sum_{j>m\in \ket{I}} 
 &+ \big( \overlap{ijm}{kjm} + \overlap{ijm}{jmk}  + \overlap{ijm}{mkj} \big) \\
 &- \big( \overlap{ijm}{jkm} + \overlap{ijm}{mjk}  + \overlap{ijm}{kmj} \big) \bigg).
 \end{aligned}
\end{equation}

\subsection{Double excitation term}
For a double excitation term, one has that 
\begin{equation}
 \begin{aligned}
  i \ne k \\
  j \ne l \\
  m = n \\
 \end{aligned}
\end{equation}
and then 
\begin{equation}
 \begin{aligned}
 \bra{I}O^3 \oprd{k}\oprd{l} \opr{j} \opr{i} \ket{I} = (-1)^{\mathcal{P}_{IJ}}
 \sum_{m\in \ket{I}} 
 &+ \big( \overlap{ijm}{klm} + \overlap{ijm}{lmk}  + \overlap{ijm}{mkl} \big) \\
 &- \big( \overlap{ijm}{lkm} + \overlap{ijm}{mlk}  + \overlap{ijm}{kml} \big) \bigg).
 \end{aligned}
\end{equation}


\end{document}
