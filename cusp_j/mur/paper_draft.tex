
\documentclass[aip,jcp,reprint,noshowkeys,superscriptaddress]{revtex4-1}
\usepackage{graphicx,dcolumn,bm,xcolor,microtype,multirow,amsmath,amssymb,amsfonts,physics,mhchem,xspace,subfigure}

\usepackage[utf8]{inputenc}
\usepackage[T1]{fontenc}
\usepackage{txfonts}

\usepackage[
	colorlinks=true,
    citecolor=blue,
    breaklinks=true
	]{hyperref}
\urlstyle{same}

\definecolor{darkgreen}{HTML}{009900}
\usepackage[normalem]{ulem}
\newcommand{\sphi}[1]{\hat{{\bf S}}_{#1}}
\newcommand{\overlap}[2]{\langle #1 | #2 \rangle}
\newcommand{\matelem}[3]{\langle #1 | #2 | #3 \rangle}
\newcommand{\deriv}[3]{\frac{\partial^{#3} #1}{\partial {#2}^{#3}}}
\newcommand{\bd}[1]{{\bf {#1}}}
\newcommand{\br}[0]{{\bf {r}}}
\newcommand{\bri}[1]{{\bf r}_{#1}}
\newcommand{\bs}[0]{{\bf {s}}}
\newcommand{\dr}[1]{\text{d}{\bf r_{#1}}}
\newcommand{\psiex}[0]{\Psi^{\text{ex}}_0}
\newcommand{\phiex}[0]{\Phi^{\text{ex}}_0}
\newcommand{\phimu}[0]{\Phi^{\text{ex},\mu}_0}
\newcommand{\phimub}[0]{\Phi^{\mathcal{B},\mu}_0}
\newcommand{\xhimub}[0]{X^{\mathcal{B},\mu}_0}
\newcommand{\psimub}[0]{\Psi^{\mathcal{B},\mu}_0}
\newcommand{\phiimub}[0]{\Phi^{\mathcal{B},\mu}_i}
\newcommand{\basis}[0]{\mathcal{B}}
\newcommand{\energyex}[0]{E^{\text{ex}}}
\newcommand{\R}{\mathbb{R}}
\newcommand{\identity}{\mathds{1}}
\newcommand{\mur}[1]{\mu({\bf r_{#1}})}
\newcommand{\muueg}{\mu_{\text{UEG}}}
\newcommand{\muuegav}{\langle \mu_{\text{UEG}}\rangle}
\newcommand{\muav}{\langle \mu\rangle}
\newcommand{\mursc}{ \mu_{r_{s,c}}}
\newcommand{\murscav}{\langle \mu_{r_{s,c}}\rangle}
\newcommand{\mursclda}{\langle \mu_{r_{s,c}^{\text{UEG}}}\rangle}



\begin{document}	

\title{A new form of transcorrelated Hamiltonian inspired by range-separated DFT: II }

\author{Emmanuel Giner}
\email{emmanuel.giner@lct.jussieu.fr}

\begin{abstract}

\end{abstract}

\maketitle

\section{The general scope}
According to Eq. (2) of Ref. \onlinecite{CohLuoGutDobTewAla-JCP-19}, the similarity transformed Hamiltonian can be written as 
\begin{equation}
 \label{ht_def_g}
 e^{-\hat{\tau}} \hat{H} e^{\hat{\tau}} = H + \big[ H,\hat{\tau} \big] + \frac{1}{2}\bigg[ \big[H,\hat{\tau}\big],\hat{\tau}\bigg]
\end{equation}
where $\hat{\tau} = \sum_{i<j}u(\br{}_i,\br{}_j)$ and $\hat{H} = \sum_i -\frac{1}{2} \nabla^2_i + v(\br{}_i) + \sum_{i<j} \frac{1}{r_{ij}}$. 
This leads to the following similarity transformed Hamiltonian 
\begin{equation}
 \begin{aligned}
 \label{ht_def_gu}
 \tilde{H} & = H - \sum_i \bigg( \frac{1}{2} \nabla_i^2 \hat{\tau} + \big(\nabla_i \hat{\tau} \big) + \frac{1}{2} \big(\nabla_i \hat{\tau} \big)^2  \bigg) \\
           & = H - \sum_{i<j} \hat{K}(\bri{i},\bri{u}) - \sum_{i<j<k} \hat{L}(\bri{i},\bri{u},\bri{k}),
 \end{aligned}
\end{equation}
where the effective two- and three-body operators $\hat{K}(\bri{i},\bri{u})$ and $\hat{L}(\bri{i},\bri{u},\bri{k})$ are defined as
\begin{equation}
 \begin{aligned}
 \label{def_k}
  \hat{K}(\bri{1},\bri{2}) ) = \frac{1}{2} \bigg( &\nabla_1^2 u(\bri{1},\bri{2}) + \nabla_2^2u(\bri{1},\bri{2}) \\
                                               + &\big(\nabla_1 u(\bri{1},\bri{2}) \big) ^2 + \big(\nabla_1 u(\bri{1},\bri{2}) \big) ^2 \bigg) \\
                                               + &\nabla_1 u(\bri{1},\bri{2}) \cdot \nabla_2 + \nabla_2 u(\bri{1},\bri{2}) \cdot \nabla_1,
 \end{aligned}
\end{equation}
\begin{equation}
 \begin{aligned}
 \label{def_k}
  \hat{L}(\bri{1},\bri{2},\bri{3}) ) = & \nabla_1 u(\bri{1},\bri{2}) \cdot \nabla_1 u(\bri{1},\bri{3}) + \nabla_2 u(\bri{2},\bri{1}) \cdot \nabla_2 u(\bri{2},\bri{3})  \\
                                     + & \nabla_3 u(\bri{3},\bri{1}) \cdot \nabla_3 u(\bri{3},\bri{2}).
 \end{aligned}
\end{equation}

Here we propose to use a non-symmetric Jastrow factor $ u(r_{12},\mur{1})$ and therefore we need to compute the integrals of the $\hat{K}(\bri{1},\bri{2})$ and $\hat{L}(\bri{1},\bri{2},\bri{3})$ operators with such a Jastrow factor. 
Here, we will give the equations for a general $\mur{1}$. 
To derive properly the integrals for such a new Jastrow factor, one needs to derive the different terms involving gradients and so on. 


\subsection{Gradients of ${u}(r_{12},\mur{1})$}
A fundamental quantity is the gradient of ${u}(\bd{r_1},\bd{r_2})$ which is 
\begin{equation}
 \nabla_1 {u}(\bd{r_1},\bd{r_2}) = \deriv{}{x_1}{} {u}(\bd{r_1},\bd{r_2}) {\bf e}_{x_1} + \deriv{}{y_1}{} {u}(\bd{r_1},\bd{r_2}) {\bf e}_{y_1} + \deriv{}{z_1}{} {u}(\bd{r_1},\bd{r_2}) {\bf e}_{z_1}.
\end{equation}
Let us begin with the first term 
\begin{equation}
 \begin{aligned}
 \deriv{}{x_1}{} u(r_{12};\mur{1})=& \deriv{}{r_{12}}{} u(r_{12};\mur{1})\deriv{r_{12}}{x_1}{} \\ 
                                  +& \deriv{}{\mur{1}}{}u(r_{12};\mur{1}) \deriv{\mur{1}}{x_1}{},
 \end{aligned}
\end{equation}
but as 
\begin{equation}
 \deriv{}{\mu}{}u(r_{12};\mu) = \frac{e^{-(\mu r_{12})^2}}{2 \sqrt{\pi} \mu^2}
\end{equation}
one obtains 
\begin{equation}
 \begin{aligned}
  \label{eq:grad_mur_1}
 \deriv{}{x_1}{} u(r_{12};\mur{1}) =& \frac{1 - \text{erf}\big(\mu(\bri{1}) r_{12} \big)}{2 r_{12}}\big( x_1 - x_2) \\
                                   +&\frac{e^{-(\mu(\bri{1}) r_{12})^2}}{2 \sqrt{\pi} \mu(\bri{1})^2} \deriv{\mur{1}}{x_1}{}.  
 \end{aligned}
\end{equation}
Similarly, as by definition $\deriv{\mur{1}}{x_2}{} = 0$, one obtains that 
\begin{equation}
 \label{eq_grad_mur_r2}
 \deriv{}{x_2}{} u(r_{12};\mur{1}) = -\big(x_1 - x_2 \big)\frac{1 - \text{erf}(\mur{1} r_{12})}{2 r_{12}} .
\end{equation}

\subsection{Computation of $\bigg( \nabla_1  u(r_{12};\mur{1})\bigg)^2 + \bigg( \nabla_2  u(r_{12};\mur{1})\bigg)^2$ }
According to Eq. \eqref{eq:grad_mur_1} one has 
\begin{equation}
 \begin{aligned}
 \deriv{}{x_1}{} u(r_{12};\mur{1}) =& \frac{1 - \text{erf}\big(\mu(\bri{1}) r_{12} \big)}{2 r_{12}}\big( x_1 - x_2) \\
                                   +&\frac{e^{-(\mu(\bri{1}) r_{12})^2}}{2 \sqrt{\pi} \mu(\bri{1})^2} \deriv{\mur{1}}{x_1}{}.  
 \end{aligned}
\end{equation}
Therefore, the computation of $\big( \deriv{}{x_1}{} u(r_{12};\mur{1}) \big)^2$ yields 
\begin{equation}
 \begin{aligned}
 \bigg( \deriv{}{x_1}{} u(r_{12};\mur{1}) \bigg)^2 = & \frac{\bigg(1 - \text{erf}\big(\mu(\bri{1}) r_{12} \big)\bigg)^2}{4 \big(r_{12}\big)^2}\big( x_1 - x_2)^2 \\
 + & \frac{e^{-2(\mu(\bri{1}) r_{12})^2}}{4 \pi \mu(\bri{1})^4} \bigg(\deriv{\mur{1}}{x_1}{} \bigg)^2 \\ 
 + & \frac{1 - \text{erf}\big(\mu(\bri{1}) r_{12} \big)}{r_{12}}\big( x_1 - x_2) \frac{e^{-(\mu(\bri{1}) r_{12})^2}}{2 \sqrt{\pi} \mu(\bri{1})^2} \deriv{\mur{1}}{x_1}{}.
 \end{aligned}
\end{equation}
Therefore, the computation of $\bigg( \nabla_1  u(r_{12};\mur{1})\bigg)^2$ yields 
\begin{equation}
 \begin{aligned}
  \bigg( \nabla_1  u(r_{12};\mur{1})\bigg)^2  = &\frac{\bigg(1 - \text{erf}\big(\mu(\bri{1}) r_{12} \big)\bigg)^2}{4} 
   +   \frac{e^{-2(\mu(\bri{1}) r_{12})^2}}{4 \pi \mu(\bri{1})^4} \bigg( \nabla_1  \mur{1}\bigg)^2 \\ 
   + & \nabla_1  \mur{1} \, \cdot \, \big( \br{}_1 - \br{}_2\big) \frac{1 - \text{erf}\big(\mu(\bri{1}) r_{12} \big)}{r_{12}} \frac{e^{-(\mu(\bri{1}) r_{12})^2}}{2 \sqrt{\pi}\mu(\bri{1})^2} 
 \end{aligned}
\end{equation}

Eventually, the total operator yields
\begin{equation}
 \label{eq:nabl_2}
 \begin{aligned}
 & \frac{\bigg( \nabla_1  u(r_{12};\mur{1})\bigg)^2 + \bigg( \nabla_2  u(r_{12};\mur{1})\bigg)^2}{2} =  \frac{\bigg(1 - \text{erf}\big(\mu(\bri{1}) r_{12} \big)\bigg)^2}{4} \\
 &  +  \frac{1}{2}\frac{e^{-2(\mu(\bri{1}) r_{12})^2}}{4 \pi \mu(\bri{1})^4} \bigg( \nabla_1  \mur{1}\bigg)^2 \\ 
 &  +  \frac{1}{2} \nabla_1  \mur{1} \, \cdot \, \big( \br{}_1 - \br{}_2\big) \frac{1 - \text{erf}\big(\mu(\bri{1}) r_{12} \big)}{r_{12}} \frac{e^{-(\mu(\bri{1}) r_{12})^2}}{2 \sqrt{\pi}\mu(\bri{1})^2} .
 \end{aligned}
\end{equation}
Note the $\frac{1}{2}$ factor in the second and third lines of Eq. \eqref{eq:nabl_2} which is due to the fact that $\mur{1}$ depends only on $\bd{r}_1$. 
These integrals remain still possible through an analytical integration on $\bd{r}_2$ and numerical integration on $\bd{r}_1$. 

\subsubsection{Computation of $\nabla_1^2  u(r_{12};\mur{1})$ }
The computation of the analytical form of the Laplacian is rather tedious, so therefore we propose to compute these integrals using the integration by part using 
\begin{equation}
 \begin{aligned}
&  \matelem{kl}{\nabla^2_1 u(r_{12};\mur{1})}{ij} \\ 
= & \int \text{d}\br_{1} \text{d}\br_{2}   \phi_l(\br_2) \phi_j(\br_2) \nabla^2_1 u(r_{12};\mur{1}) \phi_i(\br_1) \phi_k(\br_1) \\
= & -\int \text{d}\br_{1} \text{d}\br_{2}  \phi_l(\br_2) \phi_j(\br_2) \nabla_1 u(r_{12};\mur{1}) \\ & \cdot \big[ \phi_i(\br_1) \nabla_1 \phi_k(\br_1) +  \phi_k(\br_1) \nabla_1 \phi_i(\br_1)\big].
 \end{aligned}
\end{equation}
Therefore one has 
\begin{equation}
 \label{eq:lapl_1}
 \begin{aligned}
&  \matelem{kl}{\nabla^2_1 u(r_{12};\mur{1})}{ij} \\ 
= & -\int \text{d}\br_{1} \text{d}\br_{2}  \phi_l(\br_2) \phi_j(\br_2) \nabla_1 u(r_{12};\mur{1}) \\ & \cdot \big[ \phi_i(\br_1) \nabla_1 \phi_k(\br_1) +  \phi_k(\br_1) \nabla_1 \phi_i(\br_1)\big].
 \end{aligned}
\end{equation}
Then one has also to compute the same quantity for $\br_2$:
\begin{equation}
 \label{eq:lapl_2}
 \begin{aligned}
&  \matelem{kl}{\nabla^2_2 u(r_{12};\mur{1})}{ij} \\ 
= & -\int \text{d}\br_{1} \text{d}\br_{2} \phi_i(\br_1) \phi_k(\br_1) \nabla_2 u(r_{12};\mur{1}) \\ & \cdot \big[\phi_l(\br_2)  \nabla_2  \phi_j(\br_2) + \phi_j(\br_2)  \nabla_2  \phi_l(\br_2)   \big].
 \end{aligned}
\end{equation}

\subsection{Computation of the non hermitian term}
The two integrals involved in the non hermitian term are the following: 
\begin{equation}
 \label{eq:nh_1}
 \begin{aligned}
&   \matelem{kl}{ \nabla_1 u(\bri{1},\bri{2}) \cdot \nabla_2 }{ij} \\ 
=& \int \text{d}\br_{1} \text{d}\br_{2} \phi_k(\br_1) \phi_l(\br_2) \phi_i(\br_1) \nabla_1 u(\bri{1},\bri{2}) \cdot \nabla_2 \phi_j(\br_2), 
 \end{aligned}
\end{equation}
and 
\begin{equation}
 \label{eq:nh_2}
 \begin{aligned}
&   \matelem{kl}{ \nabla_2 u(\bri{1},\bri{2}) \cdot \nabla_1 }{ij} \\ 
=& \int \text{d}\br_{1} \text{d}\br_{2} \phi_k(\br_1) \phi_l(\br_2) \phi_j(\br_2) \nabla_2 u(\bri{1},\bri{2}) \cdot \nabla_1 \phi_i(\br_1). 
 \end{aligned}
\end{equation}

\subsection{Integrals to be computed for the Laplacian and non hermitian term}
\subsubsection{Sum of all terms}
Regarding the two-body part of the effective TC operator $\hat{K}(\bri{1},\bri{2}) ) $, one can notice that there are some terms which cancel each other between Eq. \eqref{eq:lapl_1}, \eqref{eq:lapl_2} and \eqref{eq:nh_1}, \eqref{eq:nh_2}. 
From Eqs. \eqref{eq:grad_mur_1} and \eqref{eq_grad_mur_r2}, one can write that
\begin{equation}
 \nabla_1 u(r_{12};\mur{1})  = - \nabla_2 u(r_{12};\mur{1}) + {\bf {\gamma}}(\br_1, r_{12}), 
\end{equation}
with 
\begin{equation}
 {\bf {\gamma}}(\br_1, r_{12}) = \frac{e^{-(\mu(\bri{1}) r_{12})^2}}{2 \sqrt{\pi} \mu(\bri{1})^2} \nabla_1{\mur{1}}
\end{equation}
Taking the chemist notation that
\begin{equation}
 \big(jl|f(\br_1,\br_2) \big) = \int \text{d}\br_{1} \text{d}\br_{2} \phi_l(\br_2) \phi_j(\br_2) f(\br_1,\br_2) 
\end{equation}
\begin{equation}
 \begin{aligned}
& \matelem{kl}{\frac{1}{2}\bigg( \nabla_1^2 u(\bri{1},\bri{2}) + \nabla_2^2u(\bri{1},\bri{2}) + \nabla_1 u(\bri{1},\bri{2}) \cdot \nabla_2 + \nabla_2 u(\bri{1},\bri{2}) \cdot \nabla_1 \bigg)}{ij} \\
 =& -\big(jl| \nabla_1 u(r_{12};\mur{1}) \big[ \phi_k(\br_1) \nabla_1 \phi_i(\br_1) + \phi_i(\br_1)\nabla_1 \phi_k(\br_1)\big] \big) \\
 &  +\big(jl| \nabla_2 u(r_{12};\mur{1}  \,\, \,\,      \phi_k(\br_1) \nabla_1 \phi_i(\br_1) \\
 &  -\big(ik| \nabla_2 u(r_{12};\mur{1}) \big[ \phi_l(\br_2) \nabla_2 \phi_j(\br_2) + \phi_j(\br_2)\nabla_2 \phi_l(\br_2)\big] \big) \\
 &  +\big(ik| \nabla_1 u(r_{12};\mur{1}) \,\, \,\,      \phi_l(\br_2) \nabla_2 \phi_j(\br_2) \\
 =& -2     \big(ik|\nabla_2u(r_{12};\mur{1}) \phi_l(\br_2) \nabla_2 \phi_j(\br_2)\big) \\
  & -\,\,\,\big(ik|\nabla_2u(r_{12};\mur{1}) \phi_j(\br_2)\nabla_2 \phi_l(\br_2) \big) \\
  & +       \big(ik|{\bf {\gamma}}(\br_1, r_{12}) \big[ \phi_l(\br_2)\nabla_2 \phi_j(\br_2) + \phi_j(\br_2)\nabla_2 \phi_l(\br_2) \big] \big) \\
  & +2      \big(jl|\nabla_2 u(r_{12};\mur{1}) \phi_k(\br_1) \nabla_1 \phi_i(\br_1) \big) \\
  & + \,\,\,\big(jl|\nabla_2 u(r_{12};\mur{1}) \phi_i(\br_1) \nabla_1 \phi_k(\br_1) \big) \\
  & - \,\,\,\big(jl|{\bf {\gamma}}(\br_1, r_{12}) \big[ \phi_k(\br_1) \nabla_1 \phi_i(\br_1) + \phi_i(\br_1) \nabla_1 \phi_k(\br_1) \big] \big)
 \end{aligned}
\end{equation}
or in a more compact form 
\begin{equation}
 \label{eq:lapl_nh_0}
 \begin{aligned}
& \matelem{kl}{\frac{1}{2}\bigg( \nabla_1^2 u(\bri{1},\bri{2}) + \nabla_2^2u(\bri{1},\bri{2}) + \nabla_1 u(\bri{1},\bri{2}) \cdot \nabla_2 + \nabla_2 u(\bri{1},\bri{2}) \cdot \nabla_1 \bigg)}{ij} \\
=& -\frac{1}{2} \big(ik|\nabla_2u(r_{12};\mur{1}) \big[ \phi_j(\br_2)\nabla_2 \phi_l(\br_2) + 2 \phi_l(\br_2) \nabla_2 \phi_j(\br_2)\big] \big) \\
 & +\frac{1}{2} \big(jl|\nabla_2u(r_{12};\mur{1}) \big[ \phi_i(\br_1)\nabla_1 \phi_k(\br_2) + 2 \phi_k(\br_1) \nabla_1 \phi_i(\br_1)\big] \big) \\
 & +\frac{1}{2} \big(ik|{\bf {\gamma}}(\br_1, r_{12}) \big[ \phi_l(\br_2)\nabla_2 \phi_j(\br_2) + \phi_j(\br_2)\nabla_2 \phi_l(\br_2) \big] \big) \\
 & -\frac{1}{2} \big(jl|{\bf {\gamma}}(\br_1, r_{12}) \big[ \phi_k(\br_1) \nabla_1 \phi_i(\br_1) + \phi_i(\br_1) \nabla_1 \phi_k(\br_1) \big] \big).
 \end{aligned}
\end{equation}

\subsubsection{The two basic types of integrals}
The integrals in Eq. \eqref{eq:lapl_nh_0} can be obtained from mixed numerical analytical integration. 
Let us take for instance the first integral of Eq. \eqref{eq:lapl_nh_0}:
\begin{equation}
 \label{eq:lapl_nh_1}
 \begin{aligned}
 &\big(ik|\nabla_2 u(r_{12};\mur{1}) \big[ \phi_j(\br_2)\nabla_2 \phi_l(\br_2) + 2 \phi_l(\br_2) \nabla_2 \phi_j(\br_2)\big] \big) \\
 &= -\int \text{d}\br_{1} \text{d}\br_{2} \frac{1 - \text{erf}(\mur{1} r_{12})}{2 r_{12}} \big(\br_1 - \br_2 \big) \cdot\\
 &   \phi_i(\br_1) \phi_k(\br_1) \big[ \phi_j(\br_2)\nabla_2 \phi_l(\br_2) + 2 \phi_l(\br_2) \nabla_2 \phi_j(\br_2)\big] .
 \end{aligned}
\end{equation}
Such an integral can be decomposed as the sum of the three components of the scalar product 
\begin{equation}
 \label{eq:lapl_nh_2}
 \begin{aligned}
 &\big(ik|\nabla_2 u(r_{12};\mur{1}) \big[ \phi_j(\br_2)\nabla_2 \phi_l(\br_2) + 2 \phi_l(\br_2) \nabla_2 \phi_j(\br_2)\big] \big) \\
&=-\big( I_{ijkl}^x + I_{ijkl}^y + I_{ijkl}^z\big),
 \end{aligned}
\end{equation}
where for instance the $x$ component  of Eq. \eqref{eq:lapl_nh_2} is simply 
\begin{equation}
 \label{eq:lapl_nh_3}
 \begin{aligned}
 &I_{ijkl}^x = \int \text{d}\br_{1} \text{d}\br_{2} \frac{1 - \text{erf}(\mur{1} r_{12})}{2 r_{12}} \big(x_1 - x_2 \big) \\
 &   \phi_i(\br_1) \phi_k(\br_1) \big[ \phi_j(\br_2)\deriv{}{x_2}{} \phi_l(\br_2) + 2 \phi_l(\br_2) \deriv{}{x_2}{} \phi_j(\br_2)\big] .
 \end{aligned}
\end{equation}
By defining the following integrals
\begin{equation}
 \label{eq:u_jl}
 \mathcal{U}_{jl}^{x}(\br) = \int \text{d}\br' \frac{1 - \text{erf}(\mu (\br') |\br - \br'|)}{2 |\br - \br'|} \phi_j(\br') \deriv{}{x'}{} \phi_l(\br'),
\end{equation}
\begin{equation}
 \label{eq:ux_jl}
 \mathcal{U}_{jl}^{xx}(\br) = \int \text{d}\br' x' \frac{1 - \text{erf}(\mu (\br') |\br - \br'|)}{2 |\br - \br'|} \phi_j(\br') \deriv{}{x'}{} \phi_l(\br'),
\end{equation}
then the integral $I_{ijkl}^x $ of Eq. \eqref{eq:lapl_nh_3} becomes 
\begin{equation}
 \label{eq:lapl_nh_1}
 \begin{aligned}
 &I_{ijkl}^x = \int \text{d}\br \phi_i(\br) \phi_k(\br) \bigg[ x \mathcal{U}_{jl}^x(\br) - \mathcal{U}_{jl}^{xx}(\br) + 2 \bigg(x \mathcal{U}_{lj}^x(\br) - \mathcal{U}_{lj}^{xx}(\br) \bigg)\bigg].
 \end{aligned}
\end{equation}
Then, the second term of Eq. \eqref{eq:lapl_nh_0} can be obtained with a similar expression
\begin{equation}
 \begin{aligned}
 & \big(jl|\nabla_2u(r_{12};\mur{1}) \big[ \phi_i(\br_1)\nabla_1 \phi_k(\br_2) + 2 \phi_k(\br_1) \nabla_1 \phi_i(\br_1)\big] \big) \\
 &=-\big( J_{ijkl}^x + J_{ijkl}^y + J_{ijkl}^z \big)
 \end{aligned}
\end{equation}
where 
\begin{equation}
 \begin{aligned}
  J_{ijkl}^x = \int \text{d}\br_1 \text{d}\br_2 & \bigg[ \phi_i(\br_1)\deriv{}{x_1}{} \phi_k(\br_2) + 2 \phi_k(\br_1) \deriv{}{x_1}{} \phi_i(\br_1)\bigg] \\  
 & \frac{1 - \text{erf}(\mur{1} r_{12})}{2 r_{12}} \big(x_1 - x_2 \big) \phi_j(\br_2) \phi_l(\br_2).
 \end{aligned}
\end{equation}


\bibliography{srDFT_SC}







\section{Garbage for fully symmetric forms}
\subsection{Computation of $\bigg(\nabla_1\tilde{u}(\bd{r_1},\bd{r_2})\bigg)^2$}
Then one needs to compute 
\begin{equation}
 \begin{aligned}
& \bigg(\nabla_1\tilde{u}(\bd{r_1},\bd{r_2})\bigg)^2 + \bigg(\nabla_2\tilde{u}(\bd{r_1},\bd{r_2})\bigg)^2 = \\
& \bigg(\deriv{}{x_1}{}\tilde{u}(\bd{r_1},\bd{r_2}) \bigg)^2 + \bigg(\deriv{}{x_2}{}\tilde{u}(\bd{r_1},\bd{r_2}) \bigg)^2 +  \\
& \bigg(\deriv{}{y_1}{}\tilde{u}(\bd{r_1},\bd{r_2}) \bigg)^2 + \bigg(\deriv{}{y_2}{}\tilde{u}(\bd{r_1},\bd{r_2}) \bigg)^2 +  \\
& \bigg(\deriv{}{z_1}{}\tilde{u}(\bd{r_1},\bd{r_2}) \bigg)^2 + \bigg(\deriv{}{z_2}{}\tilde{u}(\bd{r_1},\bd{r_2}) \bigg)^2.
 \end{aligned}
\end{equation}
According to Eq.\eqref{eq:d_dx1_jtilde}, 
and defining the complementary error function as 
\begin{equation}
 \text{erfc}(x) = 1 - \text{erf}(x)
\end{equation}
one has that 
\begin{equation}
 \begin{aligned}
& \bigg(\deriv{}{x_1}{}\tilde{u}(\bd{r_1},\bd{r_2}) \bigg)^2  \\
 = & \frac{1}{4}\frac{\big( x_1 - x_2 \big)^2}{4 \big(r_{12}\big)^2} \bigg(\text{erfc}\big(\mur{1} r_{12}\big) + \text{erfc}\big(\mur{2} r_{12}\big) \bigg)^2  \\
 + & \frac{e^{-2 \,\big(\mur{1}r_{12} \big)^2}}{4\pi \big(\mur{1}\big)^4} \bigg( \deriv{}{x_1}{}\mur{1} \bigg)^2 \\
 + & \frac{\big(x_1 - x_2\big)}{r_{12}} \frac{e^{-\big(\mur{1} r_{12}\big)^2}}{2\sqrt{\pi} \big(\mur{1}\big)^2} \deriv{}{x_1}{}\mur{1} 
 \frac{\text{erfc}\big(\mur{1} r_{12}\big) + \text{erfc}\big(\mur{2} r_{12}\big)}{2}.
 \end{aligned}
\end{equation}
Similarly 
\begin{equation}
 \begin{aligned}
& \bigg(\deriv{}{x_2}{}\tilde{u}(\bd{r_1},\bd{r_2}) \bigg)^2  \\
 = & \frac{1}{4}\frac{\big( x_1 - x_2 \big)^2}{4 \big(r_{12}\big)^2} \bigg(\text{erfc}\big(\mur{1} r_{12}\big) + \text{erfc}\big(\mur{2} r_{12}\big) \bigg)^2  \\
 + & \frac{e^{-2 \,\big(\mur{2}r_{12} \big)^2}}{4\pi \big(\mur{2}\big)^4} \bigg( \deriv{}{x_2}{}\mur{2} \bigg)^2 \\
 + & \frac{\big(x_2 - x_1\big)}{r_{12}} \frac{e^{-\big(\mur{2} r_{12}\big)^2}}{2\sqrt{\pi} \big(\mur{2}\big)^2} \deriv{}{x_2}{}\mur{2} 
 \frac{\text{erfc}\big(\mur{1} r_{12}\big) + \text{erfc}\big(\mur{2} r_{12}\big)}{2}.
 \end{aligned}
\end{equation}
Therefore, 
\begin{equation}
 \begin{aligned}
& \bigg(\deriv{}{x_1}{}\tilde{u}(\bd{r_1},\bd{r_2}) \bigg)^2 + \bigg(\deriv{}{x_2}{}\tilde{u}(\bd{r_1},\bd{r_2}) \bigg)^2  \\
 = & \frac{1}{2}\frac{\big( x_1 - x_2 \big)^2}{4 \big(r_{12}\big)^2} \bigg(\text{erfc}\big(\mur{1} r_{12}\big) + \text{erfc}\big(\mur{2} r_{12}\big) \bigg)^2  \\
 + & \frac{\big(x_1 - x_2\big)}{2 \sqrt{\pi}r_{12}}  \frac{\bigg(\text{erfc}\big(\mur{1} r_{12}\big) + \text{erfc}\big(\mur{2} r_{12}\big) \bigg)}{2} \\
&\bigg(   \frac{e^{-\big(\mur{1} r_{12}\big)^2}}{\big(\mur{1}\big)^2} \deriv{}{x_1}{}\mur{1} 
- \frac{e^{-\big(\mur{2} r_{12}\big)^2}}{\big(\mur{2}\big)^2} \deriv{}{x_2}{}\mur{2} \bigg) \\
 + & \frac{e^{-2 \,\big(\mur{1}r_{12} \big)^2}}{4\pi \big(\mur{1}\big)^4} \bigg( \deriv{}{x_1}{}\mur{1} \bigg)^2 
 + \frac{e^{-2 \,\big(\mur{2}r_{12} \big)^2}}{4\pi \big(\mur{2}\big)^4} \bigg( \deriv{}{x_2}{}\mur{2} \bigg)^2.
 \end{aligned}
\end{equation}

Summing all terms and noticing that $\bigg(x_1 - x_2\bigg)^2 + \bigg(y_1 - y_2\bigg)^2 + \bigg(z_1 - z_2\bigg)^2 = \bigg(r_{12}\bigg)^2$ leads to 
\begin{equation}
 \begin{aligned}
& \bigg(\nabla_1\tilde{u}(\bd{r_1},\bd{r_2})\bigg)^2 + \bigg(\nabla_2\tilde{u}(\bd{r_1},\bd{r_2})\bigg)^2 = 
  \frac{1}{8}\bigg(\text{erfc}\big(\mur{1} r_{12}\big) + \text{erfc}\big(\mur{2} r_{12}\big) \bigg)^2  \\
&+\frac{\text{erfc}\big(\mur{1} r_{12}\big) + \text{erfc}\big(\mur{2} r_{12}\big)}{4 \sqrt{\pi}r_{12}}  \\
& \big( \bri{1} - \bri{2}\big)\cdot
\bigg(\frac{e^{- \,\big(\mur{1}r_{12} \big)^2}}{4\pi \big(\mur{1}\big)^4} \nabla_1 \mur{1}
  -   \frac{e^{- \,\big(\mur{2}r_{12} \big)^2}}{4\pi \big(\mur{2}\big)^4} \nabla_2 \mur{2}  \bigg) \\
  &+ \frac{e^{-2 \,\big(\mur{1}r_{12} \big)^2}}{4\pi \big(\mur{1}\big)^4} \bigg( \deriv{}{x_1}{}\mur{1} \bigg)^2 
 + \frac{e^{-2 \,\big(\mur{2}r_{12} \big)^2}}{4\pi \big(\mur{2}\big)^4} \bigg( \deriv{}{x_2}{}\mur{2} \bigg)^2.
 \end{aligned}
\end{equation}

% f(x) = (3*erf(mu*x) - 1)/(2*x) + 1.5*mu/sqpi * exp(-(mu*x)**2.) - 0.25 * (1 - erf(mu*x))**2.
\subsection{Proposal of different symmetrization}
Let us assume that we only take the following form as a Jastrow factor 
\begin{equation}
 J(\br_1,\br_2) = \frac{e^{ u(r_{12};\mur{1})} + e^{u(r_{12};\mur{2})}}{2}. 
\end{equation}

\end{document}
